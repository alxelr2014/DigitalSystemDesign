\chapter{Verilog}
\lstset{language = Verilog}

Behavioral design describes circuit as an algorithm and structural design describes explicit circuit elements. A behavioral example for a full adder.
\begin{lstlisting}
    module adder(a,b,cin,s,cout);
        output s,cout;
        input a,b,cin;
        reg s, cout;
        always @(a or b or cin)
        begin
            s = a^b^cin;
            cout = a&b | a& cin | b& cin;
        end
    endmodule
\end{lstlisting}

And an example for a structural design of the same full adder.
\begin{lstlisting}
    module adder(a,b,cin,s,cout)
        output s,cout;
        input a,b,cin;

        wire w1,w2,w3,w4,w5;
        xor g1(w1,a,b);
        xor g2(s,w1,cin);
        and g3(w2,a,b);
        and g4(w3,a,cin);
        and g5(w4,b,cin);
        or g6(w5,w2,w3);
        or g7(cout,w4,w5);
    endmodule
\end{lstlisting}
Number are specified by:
\begin{itemize}
    \item Radix ('b,'h,'o,'d)
    \item Bit-length
    \item Value
\end{itemize}
for example
\begin{itemize}
    \item 4'b1011
    \item 12'habc
    \item 16'd255
\end{itemize}



\subsection{4-value logic}
The set of value are \(\{0,1,x,z\}\). \(x\) is for unknown and \(z\) is for high impedence.
Signal strengths are (in order):
\begin{enumerate}
    \item supply (Driving)
    \item strong (Driving)
    \item pull (Driving)
    \item large (Storage)
    \item weak (Driving)
    \item medium (Storage)
    \item small (Storage)
    \item highz (High impedence)
\end{enumerate}

wire is used to represent connections between hardware elements (default = z).
reg for hardware as well but it retains its value until next assignment (default = x).

\begin{enumerate}
    \item Syntax

          \begin{lstlisting}
    wire/reg [msb_index : lsb_index] data_id;
\end{lstlisting}
    \item Example
          \begin{lstlisting}
        wire a;
        wire [7:0] bus;
        wire [31:0] busA, busB, busC;
        reg clock;
        reg [0:40] virtual_addr;
    \end{lstlisting}
\end{enumerate}

modules are models of hardware. internals not visible to enviroment. internals can be changed as long as the interface (ports) is not changed.
\begin{lstlisting}
    module fulladd4 (sum, c_out,a,b,c_in)
    ... 
    endmodule  
\end{lstlisting}
Ports are the terminals (pins) which have three types (in, out, inout). For port declaration
\begin{lstlisting}
    input a,b,sel;
    input signed [15:0] a,b;
    output signed [31:9] result;
    output reg signed [32:1] sun;
    inout [15:12] addr;
\end{lstlisting}
In a module, inputs are always of type net but output can be reg as well. inout can be of type net only.
We can implement delays as such
\begin{lstlisting}
    and #(delay-time) a1 (out,i1,i2);
    and #(rise-val,fall-val) a1 (out,i1,i2);
    and #(min:typ:max,min:typ:max) a1 (out,i1,i2);
    bufif0 #(rise-val,fall-val,turnoff-val) a1 (out,in,control);
\end{lstlisting}